% Copyright (c) 2005-2007 Nokia Corporation
%
% Licensed under the Apache License, Version 2.0 (the "License");
% you may not use this file except in compliance with the License.
% You may obtain a copy of the License at
%
%     http://www.apache.org/licenses/LICENSE-2.0
%
% Unless required by applicable law or agreed to in writing, software
% distributed under the License is distributed on an "AS IS" BASIS,
% WITHOUT WARRANTIES OR CONDITIONS OF ANY KIND, either express or implied.
% See the License for the specific language governing permissions and
% limitations under the License.

\chapter{API Summary}
\label{sec:summary}

All built-in object types of the Python language are supported in the
S60 environment. The rest of the programming interfaces are
implemented by various library modules as summarized in this chapter.

\section{Python Standard Library}
\label{subsec:python}

Python for S60 platform distribution does not include all of the 
Python's standard and optional library modules to save storage space in the 
phone. Nevertheless, many of the excluded modules also work in the S60 
Python environment without any modifications. Some modules are included in 
the SDK version but not installed in the phone. For a summary of supported 
library modules, see Chapter \ref{s60lib}.

When Python, available at \url{http://www.python.org/}, is installed on a PC, the 
library modules are by default located in \file{\textbackslash Python22\textbackslash Lib}
on Windows and in \file{/usr/lib/python2.2} on Linux. The Python library 
modules' APIs are documented in \cite{PyLibRef}.

Python for S60 extends some standard modules. These extensions are 
described in this document, see Chapter \ref{extensions}.

\section{Python for S60 Extensions}
\label{sec:sumext}

There are two kinds of native C++ extensions in the Python for S60 
Platform: built-in extensions and dynamically loadable extensions.

\subsection{Built-in extensions}
\label{sec:built}

There are two built-in extensions in the Python for S60 package:

\begin{itemize}
\item The \refmodule{e32} extension module is built into the Python interpreter on Symbian OS, and implements interfaces to special Symbian OS Platform services that are not accessible via Python standard library modules.
\item The \refmodule{appuifw} module for Python for S60 Platform offers UI application framework related Python interfaces.
\end{itemize}

\subsection{Dynamically loadable extensions}
\label{sec:dynamically}

These dynamically loadable extension modules provide proprietary APIs
to S60 Platform's services: 
\begin{itemize}
\item \mbox{\refmodule{graphics}}: see Chapter \ref{sec:graphics}
\item \mbox{\refmodule{e32db}}: see Chapter \ref{sec:e32db}
\item \mbox{\refmodule{messaging}}: see Chapter \ref{sec:messaging}
\item \mbox{\refmodule{inbox}}: see Chapter \ref{sec:inbox}
\item \mbox{\refmodule{location}}: see Chapter \ref{sec:location}
\item \mbox{\refmodule{sysinfo}}: see Chapter \ref{sec:sysinfo}
\item \mbox{\refmodule{camera}}: see Chapter \ref{sec:camera}
\item \mbox{\refmodule{audio}}: see Chapter \ref{sec:audio}
\item \mbox{\refmodule{telephone}}: see Chapter \ref{sec:telephone}
\item \mbox{\refmodule{calendar}}: see Chapter \ref{sec:calendar}
\item \mbox{\refmodule{contacts}}: see Chapter \ref{sec:contacts}
\item \mbox{\refmodule{keycapture}}: see Chapter \ref{sec:keycapture}
\item \mbox{\refmodule{topwindow}}: see Chapter \ref{sec:topwindow}
\item \mbox{\refmodule{gles}}: see Chapter \ref{sec:gles}
\item \mbox{\refmodule{glcanvas}}: see Chapter \ref{sec:glcanvas}
\end{itemize}

\section{Third-Party Extensions}
\label{subsec:third}

% XXX appendix references
It is also possible to write your own Python extensions. S60 related
extensions to Python/C API are described in Chapter
\ref{capiextensions}. For some further guidelines on writing
extensions in C/C++, see Chapter \ref{extending}. In
addition, for an example on porting a simple extension to S60, see
\cite{PyS60Prog}.
