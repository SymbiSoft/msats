% Copyright (c) 2005 - 2007 Nokia Corporation
%
% Licensed under the Apache License, Version 2.0 (the "License");
% you may not use this file except in compliance with the License.
% You may obtain a copy of the License at
%
%     http://www.apache.org/licenses/LICENSE-2.0
%
% Unless required by applicable law or agreed to in writing, software
% distributed under the License is distributed on an "AS IS" BASIS,
% WITHOUT WARRANTIES OR CONDITIONS OF ANY KIND, either express or implied.
% See the License for the specific language governing permissions and
% limitations under the License.

\section{Extensions to Standard Library Modules}
\label{extensions}

The following standard modules have been extended.

\subsection{\module{thread} ---
  S60 extensions to standard thread module} 
\label{subsec:thread}

\declaremodule{extension}{thread}
\modulesynopsis{S60 extensions to standard thread module.}

The following function has been added to the standard \code{thread} 
module:

\begin{funcdesc}{ao_waittid}{thread_id}

Synchronizes with the end of the execution of the thread identified by the given 
\var{thread_id}. The implementation is based on a Symbian OS active object. 
For the blocking behavior, see Section \ref{subsec:Aolock}, Ao_lock Type.

\end{funcdesc}

\subsection{\module{socket} ---
  S60 extensions to standard socket module} 
\label{subsec:socket}

\declaremodule{extension}{socket}
\modulesynopsis{Extensions to standard socket module.}

Bluetooth (BT) support has been added to the standard \code{socket} 
module. The following related constants and functions are defined:

\begin{notice}[note]
In release 1.0 the functions \code{bt_advertise_service}, 
\code{bt_obex_receive}, and 
\code{bt_rfcomm_get_available_server_channel} incorrectly 
expected to be given the internal \code{e32socket.socket} object as the 
socket parameter instead of the proper \code{socket} object. Now the 
functions work correctly. The old calling convention is still supported but 
it is deprecated and may be removed in a future release.
\end{notice}

\begin{datadesc}{AF_BT}

Represents the Bluetooth address family.

\end{datadesc}

\begin{datadesc}{BTPROTO_RFCOMM}

This constant represents the Bluetooth protocol RFCOMM.

\end{datadesc}

\begin{datadesc}{RFCOMM}
\end{datadesc}
\begin{datadesc}{OBEX}

Bluetooth service classes supported by \code{bt_advertise_service}.

\end{datadesc}

\begin{datadesc}{AUTH}
\end{datadesc}
\begin{datadesc}{ENCRYPT}
\end{datadesc}
\begin{datadesc}{AUTHOR}

Bluetooth security mode flags.

\end{datadesc}

\begin{funcdesc}{bt_advertise_service}{name, socket, flag, class}

Sets a service advertising the service \var{name} (Unicode) on local channel 
that is bound to \var{socket}. If \var{flag} is \code{True}, the advertising is 
turned on, otherwise it is turned off. The service class to be advertised is 
either \code{RFCOMM} or \code{OBEX}.

\end{funcdesc}

\begin{funcdesc}{bt_discover}{\optional{address}}

Performs the Bluetooth device discovery (if the optional BT device address 
is not given) and the discovery of RFCOMM class services on the chosen 
device. Returns a pair: BT device address, dictionary of services, where 
Unicode service name is the key and the corresponding port is the value.

\end{funcdesc}

\begin{funcdesc}{bt_obex_discover}{\optional{address}}

Same as \code{discover}, but for discovery of OBEX class services on the 
chosen device.

\end{funcdesc}

\begin{funcdesc}{bt_obex_send_file}{address, channel, filename}

Sends file \var{filename} (Unicode) wrapped into an OBEX object 
to remote \var{address}, \var{channel}.

\end{funcdesc}

\begin{funcdesc}{bt_obex_receive}{socket, filename}

Receives a file as an OBEX object, unwraps and stores it into \var{filename} 
(Unicode). \var{socket} is a bound \code{OBEX} socket.

\end{funcdesc}

\begin{funcdesc}{bt_rfcomm_get_available_server_channel}{socket}

Returns an available RFCOMM server channel for \var{socket}.

\end{funcdesc}

\begin{funcdesc}{set_security}{socket, mode}

Sets the security level of the given bound \var{socket}. The 
\var{mode} is an integer flag that is formed using a binary 
\code{or} operation of one or more of: \code{AUTH} (authentication), 
\code{ENCRYPT}, \code{AUTHOR} (authorization). Example: 
\code{set_security(s, AUTH | AUTHOR)}.

\end{funcdesc}

\begin{notice}[note]
When listening to a Bluetooth socket on the phone, it is necessary to set 
the security level.
\end{notice}

\begin{notice}[note]
SSL is not supported in S60 1st Edition. SSL client certificates are 
not supported at all.
\end{notice}

For examples on the usage of these functions, see Programming with Python for 
S60 Platform \cite{PyS60Prog}.

Setting default Access Point (AP) has been added to the standard \code{socket} 
module. The following related constants and functions are defined:

\begin{funcdesc}{select_access_point}{}
This opens popup selection where access points are listed and can be selected.
Returns selected access point id.
\end{funcdesc}

\begin{funcdesc}{access_point}{apid}
This creates access point object by given apid. Returns access point object.
\end{funcdesc}

\begin{funcdesc}{set_default_access_point}{apo}
This sets the default access point that is used when socket is opened. Setting 
\var{apo} to \code{"None"} will clear default access point.
\end{funcdesc}

\begin{funcdesc}{access_points}{}
This lists access points id's and names that are available. 
\end{funcdesc}

Example 1:
\begin{verbatim}
#access point is selected from the list
apid = select_access_point()
apo = access_point(apid)
set_default_access_point(apo)

s = socket(AF_INET, SOCK_STREAM)
print apo.ip()
s.connect(('www.sourceforge.net',80))
s.send('GET /\r\n\r\n')
s.recv(100)
s.close()
apo.stop()

\end{verbatim}

Example 2:
\begin{verbatim}
#Access point id is already known
apo = access_point(1)
set_default_access_point(apo) 

s = socket(AF_INET, SOCK_STREAM)
s.connect(('www.sourceforge.net',80))
s.send('GET /\r\n\r\n')
s.recv(100)
s.close()
apo.stop()
\end{verbatim}

Example 3:
\begin{verbatim}
#display interface ip.
#access point is selected from the list
apid = select_access_point()
apo = access_point(apid)
apo.start()
#Note that ip-address is given by operator, if static ip-address is not defined,
#when connection is started
print apo.ip()
#When connection is closed dynamic ip-address is released
apo.stop()
\end{verbatim}

