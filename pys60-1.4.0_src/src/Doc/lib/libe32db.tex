% Copyright (c) 2005 Nokia Corporation
%
% Licensed under the Apache License, Version 2.0 (the "License");
% you may not use this file except in compliance with the License.
% You may obtain a copy of the License at
%
%     http://www.apache.org/licenses/LICENSE-2.0
%
% Unless required by applicable law or agreed to in writing, software
% distributed under the License is distributed on an "AS IS" BASIS,
% WITHOUT WARRANTIES OR CONDITIONS OF ANY KIND, either express or implied.
% See the License for the specific language governing permissions and
% limitations under the License.

\section{\module{e32db} ---
  Interface to the Symbian native DB}
\label{sec:e32db}

\declaremodule{extension}{e32db}
\platform{S60}
\modulesynopsis{Interface to the Symbian native DB}

\label{sec:mylabel3}
The \module{e32db} module provides an API for relational database 
manipulation with a restricted SQL syntax. For details of DBMS support, see 
the S60 SDK documentation. For examples on using this module, see \cite{PyS60Prog}.

The \module{e32db} module defines the following functions:

\begin{funcdesc}{format_rawtime}{timevalue}
Formats \var{timevalue} (Symbian time) according to the current 
system's date/time formatting rules and returns it as a Unicode string.
\end{funcdesc}

\begin{funcdesc}{format_time}{timevalue}
Formats \var{timevalue} according to the current system's date/time 
formatting rules and returns it as a Unicode string.
\end{funcdesc}

\subsection{Dbms Objects}
\label{subsec:mylabel13}

\begin{classdesc}{Dbms}{}
Creates a Dbms object. Dbms objects support basic 
operations on a database. 
\end{classdesc}

Dbms objects have the following methods:

\begin{methoddesc}[Dbms]{begin}{}
Begins a transaction on the database.
\end{methoddesc}

\begin{methoddesc}[Dbms]{close}{}
Closes the database object. It is safe to try to close a database object 
even if it is not open.
\end{methoddesc}

\begin{methoddesc}[Dbms]{commit}{}
Commits the current transaction.
\end{methoddesc}

\begin{methoddesc}[Dbms]{compact}{}
Compacts the database, reclaiming unused space in the database file. 
\end{methoddesc}

\begin{methoddesc}[Dbms]{create}{dbname}
Creates a database with path \var{dbname}.
\end{methoddesc}

\begin{methoddesc}[Dbms]{execute}{query}
Executes an SQL \var{query}. On success, returns \code{0} if a DDL
(SQL schema update) statement was executed. Returns the number of rows
inserted, updated, or deleted, if a DML (SQL data update) statement
was executed.
\end{methoddesc}

\begin{methoddesc}[Dbms]{open}{dbname}
Opens the database in file \var{dbname}. This should be a full 
Unicode path name, for example, \code{u'c:\e\e foo.db'}.
\end{methoddesc}

\begin{methoddesc}[Dbms]{rollback}{}
Rolls back the current transaction.
\end{methoddesc}

\subsection{DB_view Objects}
\label{subsec:mylabel14}

\begin{classdesc}{Db_view}{}
Creates a \class{Db_view} object. \class{DB_view} objects generate 
rowsets from a SQL query. They provide functions to parse and evaluate the 
rowsets.
\end{classdesc}

Db_view objects have the following methods:

\begin{methoddesc}[Db_view]{col}{column}
Returns the value in \var{column}. The first column of the rowset has the index 
\code{1}. If the type of the column is not supported, a \exception{TypeError} is 
raised. See Table \ref{tab:sqltypes} for a list of supported data types.
\end{methoddesc}

\begin{methoddesc}[Db_view]{col_count}{}
Returns the number of columns defined in the rowset.
\end{methoddesc}

\begin{methoddesc}[Db_view]{col_length}{column}
Gets the length of the value in \var{column}. Empty columns have 
a length of zero; non-empty numerical and date/time columns have a length of 
1. For text columns, the length is the character count, and for binary 
columns, the length is the byte count.
\end{methoddesc}

\begin{methoddesc}[Db_view]{col_raw}{column}
Extracts the value of \var{column} as raw binary data, and 
returns it as a Python string. The first column of the rowset has the index 
1. See Table \ref{tab:sqltypes} for a list of supported data types.
\end{methoddesc}

\begin{methoddesc}[Db_view]{col_rawtime}{column}
Extracts the value of a date/time column at index \var{column} as a
long integer, which represents the raw Symbian time value. The first
column of the rowset has the index 1.  See Table \ref{tab:sqltypes} for a list of the
supported data types.
\end{methoddesc}

\begin{methoddesc}[Db_view]{col_type}{column}
Returns the numeric type of the given column as an integer from a 
Symbian-specific list of types. This function is used in the implementation 
of method \method{col}.
\end{methoddesc}

\begin{methoddesc}[Db_view]{count_line}{}
Returns the number of rows available in the rowset.
\end{methoddesc}

\begin{methoddesc}[Db_view]{first_line}{}
Positions the cursor on the first row in the rowset.
\end{methoddesc}

\begin{methoddesc}[Db_view]{get_line}{}
Gets the current row data for access.
\end{methoddesc}

\begin{methoddesc}[Db_view]{is_col_null}{column}
Tests whether \var{column} is empty. Empty columns can be 
accessed like normal columns. Empty numerical columns return a \code{0} or 
an equivalent value, and text and binary columns have a zero length.
\end{methoddesc}

\begin{methoddesc}[Db_view]{next_line}{}
Moves the cursor to the next row in the rowset.
\end{methoddesc}

\begin{methoddesc}[Db_view]{prepare}{db, query}
Prepares the view object for evaluating an SQL select statement. 
\var{db} is a \class{Dbms} object and \var{query}
the SQL query to be executed.
\end{methoddesc}

\subsection{Mapping Between SQL and Python Data Types }
\label{subsec:mapping}
See Table \ref{tab:sqltypes} for a summary of mapping between SQL and 
Python data types. The \method{col} function can extract any value except 
\code{LONG VARBINARY} and return it as the proper Python value. In 
addition, the \method{col_raw} function can extract any column type 
except \code{LONG VARCHAR} and \code{LONG VARBINARY} as raw binary data 
and return it as a Python string.

Inserting, updating, or searching for \code{BINARY}, \code{VARBINARY}, 
or \code{LONG VARBINARY} values is not supported. \code{BINARY} and 
\code{VARBINARY} values can be read with \method{col} or 
\method{col_raw}.

\begin{table}[htbp]
\begin{center}
\begin{tabular}{|p{117pt}|p{144pt}|p{99pt}|p{63pt}|}
\hline
SQL type& 
Symbian column type (in the DBMS C++ API)& 
Python type& 
Supported \\
\hline
\textsf{BIT}& 
\textsf{EDbColBit}& 
\raisebox{-10.50ex}[0cm][0cm]{int}& 
\raisebox{-27.00ex}[0cm][0cm]{yes} \\
\cline{1-2} 
\textsf{TINYINT}& 
\textsf{EDbColInt8}& 
 & 
  \\
\cline{1-2} 
\textsf{UNSIGNED TINYINT}& 
\textsf{EDbColUint8}& 
 & 
  \\
\cline{1-2} 
\textsf{SMALLINT}& 
\textsf{EDbColInt16}& 
 & 
  \\
\cline{1-2} 
\textsf{UNSIGNED SMALLINT}& 
\textsf{EDbColUint16}& 
 & 
  \\
\cline{1-2} 
\textsf{INTEGER}& 
\textsf{EDbColInt32}& 
 & 
  \\
\cline{1-2} 
\textsf{UNSIGNED INTEGER}& 
\textsf{EDbColUint32}& 
 & 
  \\
\cline{1-2} 
\textsf{COUNTER}& 
\textsf{EDbColUint32 (}with the\textsf{ TDbCol::EAutoIncrement }attribute\textsf{)}& 
 & 
  \\
\cline{1-3} 
\textsf{BIGINT}& 
\textsf{EDbColInt64}& 
long& 
  \\
\cline{1-3} 
\textsf{REAL}& 
\textsf{EDbColReal32}& 
\raisebox{-4.50ex}[0cm][0cm]{float  \par }& 
  \\
\cline{1-2} 
\textsf{FLOAT}& 
\raisebox{-3.00ex}[0cm][0cm]{\textsf{EDbColReal64} \par \textsf{}}& 
 & 
  \\
\cline{1-1} 
\textsf{DOUBLE}& 
 & 
 & 
  \\
\cline{1-1} 
\textsf{DOUBLE PRECISION}& 
 & 
 & 
  \\
\cline{1-3} 
\textsf{DATE}& 
\raisebox{-3.00ex}[0cm][0cm]{\textsf{EDbColDateTime} \par \textsf{}}& 
\raisebox{-3.00ex}[0cm][0cm]{float \par (or long, with \textsf{col_rawtime()})}& 
  \\
\cline{1-1} 
\textsf{TIME}& 
 & 
 & 
  \\
\cline{1-1} 
\textsf{TIMESTAMP}& 
 & 
 & 
  \\
\cline{1-3} 
\textsf{CHAR(n)}& 
\raisebox{-1.50ex}[0cm][0cm]{\textsf{EDbColText}}& 
\raisebox{-3.00ex}[0cm][0cm]{Unicode}& 
  \\
\cline{1-1} 
\textsf{VARCHAR(n)}& 
 & 
 & 
  \\
\cline{1-2} 
\textsf{LONG VARCHAR}& 
\textsf{EDbColLongText}& 
 & 
  \\
\hline
\textsf{BINARY(n)}& 
\raisebox{-1.50ex}[0cm][0cm]{\textsf{EDbColBinary} \par \textsf{}}& 
\raisebox{-1.50ex}[0cm][0cm]{str}& 
\raisebox{-1.50ex}[0cm][0cm]{read only} \\
\cline{1-1} 
\textsf{VARBINARY(n)}& 
 & 
 & 
  \\
\hline
\textsf{LONG VARBINARY}& 
\textsf{EDbColLongBinary}& 
n/a& 
no \\
\hline
\end{tabular}
\caption{Mapping between SQL and Python types}
\label{tab:sqltypes}
\end{center}
\end{table}


\subsection{Date and Time Handling}
\label{subsec:mylabel15}
The functions \method{col} and \textsf{format_time} use Unix time, 
seconds since January 1, 1970, 00:00:00 UTC, as the time format. Internally 
the database uses the native Symbian time representation that provides 
greater precision and range than the Unix time. The native Symbian time 
format is a 64-bit value that represents microseconds since January 1st 0 AD 
00:00:00 local time, nominal Gregorian. BC dates are represented by negative 
values. Since converting this format to Unix time and back may cause slight 
round-off errors, you have to use the functions \function{col_rawtime} and 
\function{format_rawtime} if you need to be able to handle these values 
with full precision.

The representation of date and time literals in SQL statements depends on 
the current system date and time format. Note that the only accepted 
ordering of day, month, and year is the one that the system is currently 
configured to use. Dates in other order are rejected. The recommended way to 
form date/time literals for SQL statements is to use the functions 
\function{format_time} or \function{format_rawtime} that format the given 
date/time values properly according to the current system's date/time format 
settings.
