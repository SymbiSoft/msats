% Copyright (c) 2005 - 2007 Nokia Corporation
%
% Licensed under the Apache License, Version 2.0 (the "License");
% you may not use this file except in compliance with the License.
% You may obtain a copy of the License at
%
%     http://www.apache.org/licenses/LICENSE-2.0
%
% Unless required by applicable law or agreed to in writing, software
% distributed under the License is distributed on an "AS IS" BASIS,
% WITHOUT WARRANTIES OR CONDITIONS OF ANY KIND, either express or implied.
% See the License for the specific language governing permissions and
% limitations under the License.

\section{\module{location} ---
	 GSM location information}
\label{sec:location}

\declaremodule{extension}{location}
\platform{S60}
\modulesynopsis{Package supporting location information fetching.}

The \module{location} module offers APIs to location information related 
services. Currently, the \module{location} has one function:

\begin{notice}[note]
Location module requires capabilities ReadDeviceData, ReadUserData and Location
in 3rd Edition devices.
\end{notice}

\begin{funcdesc}{gsm_location}{}
Retrieves GSM location information: Mobile Country Code, Mobile Network Code, 
Location Area Code, and Cell ID. A location area normally consists of several 
base stations. It is the area where the terminal can move without notifying the 
network about its exact position. mcc and mnc together form a unique 
identification number of the network into which the phone is logged.
\end{funcdesc}

\subsection{Examples}

Here is an example of how to use the \module{location} package to
fetch the location information:

\begin{verbatim}
>>> import location
>>> print location.gsm_location()
\end{verbatim}
