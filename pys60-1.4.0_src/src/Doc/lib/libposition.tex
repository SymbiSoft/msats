% Copyright (c) 2007 Nokia Corporation
%
% Licensed under the Apache License, Version 2.0 (the "License");
% you may not use this file except in compliance with the License.
% You may obtain a copy of the License at
%
%     http://www.apache.org/licenses/LICENSE-2.0
%
% Unless required by applicable law or agreed to in writing, software
% distributed under the License is distributed on an "AS IS" BASIS,
% WITHOUT WARRANTIES OR CONDITIONS OF ANY KIND, either express or implied.
% See the License for the specific language governing permissions and
% limitations under the License.


\section{\module{position} ---
         Simplified interface to the Location Acquisition API}
\label{sec:position}

\declaremodule{extension}{position}		% not standard, in C
\platform{S60}
\modulesynopsis{Simplified interface to the Location Acquisition API.}

The \module{position} provides basic access to the S60 Location Acquisition API. 
The Location Acquisition API gathers different positioning technologies together to be used 
through a consistent interface. Location Acquisition API offers quite a large 
amount of information (cost of service, device power consumption etc.) about 
accessible positioning devices (like GPS-modules), position, course, accuracy 
and satellite information (depending on the position device used) and much 
more. The Location Acquisition API can also be used to obtain device/vendor 
specific extended information.

The \module{position} module has the following functions:

\begin{funcdesc}{modules}{}
get information about available positioning modules
\end{funcdesc}

\begin{funcdesc}{default_module}{}
get default module id
\end{funcdesc}

\begin{funcdesc}{module_info}{module_id}
get detailed information about the specified module
\end{funcdesc}

\begin{funcdesc}{select_module}{module_id}
select a module
\end{funcdesc}

\begin{funcdesc}{set_requestors}{requestors}
set the \var{requestors} of the service (at least one must be set)
\end{funcdesc}

\begin{funcdesc}{position}{course=0,satellites=0}
get the position information
\end{funcdesc}


\subsection{Example \label{position-example}}

The following example demonstrates how to use the python \module{position} module.

\begin{verbatim}
# information about available positioning modules
print "***available modules***"
print positioning.modules()
print ""

# id of the default positioning module
print "***default module***"
print positioning.default_module()
print ""

# detailed information about the default positioning module
print "***detailed module info***"
print positioning.module_info(positioning.default_module())
print ""

# select a module (however, selecting default module has no 
# relevance..this has been added just to show the functionality). 
positioning.select_module(positioning.default_module())

# set requestors.
# at least one requestor must be set before requesting the position.
# the last requestor must always be service requestor 
# (whether or not there are other requestors). 
positioning.set_requestors([{"type":"service",
                             "format":"application",
                             "data":"test_app"}])

# get the position. 
# note that the first position()-call may take a long time
# (because of gps technology).
print "***position info***"                         
print positioning.position()
print ""

# re-get the position.
# this call should be much quicker.
# ask also course and satellite information.
print "***course and satellites***" 
print positioning.position(course=1,satellites=1)
print ""
\end{verbatim}

To run the script in the emulator you must configure PSY emulation 
(SimPSYConfigurator-\textgreater Select Config File -\textgreater \textless 
some config file s\textgreater).
